\documentclass[german,plainarticle,hyperref,utf8]{zihpub}
\author{Daniel Körsten}
\title{Komplexpraktikum Paralleles Rechnen}
\matno{4690396}
\betreuer{Dr. Robert Schöne}
\bibfiles{bib-filenames}
\begin{document}
	\section{Aufgabe B}
	\subsection{Beschreibung}
	In dieser Aufgabe wird Thread-parallele Ausführung von \verb|Conways Game-of-Life| durchgeführt. Conways Game-of-Life kann man sich als $n\times m$ Matrix vorstellen, bei der in jedem Berechnungsschritt die nächste Generation berechnet wird. Die Spielregeln lassen sich \href{https://de.wikipedia.org/wiki/Conways_Spiel_des_Lebens#Die_Spielregeln}{hier} nachlesen.
	
	\subsection{Vorbereitung des Codes für eine parallele Ausführung}
	Für die Berechnung der nächsten Generation muss nun jede Zelle einzeln betrachtet werden und ihr Zustand in der nächsten Generation gemäß den Spielregeln berechnet werden.
	Ein möglicher Ansatz ist über jede Zeile und anschließend jede Spalte zu iterieren. Realisieren lässt sich das über zwei geschachtelte \verb|for|-Schleifen. Das Ergebnis dieser Berechnung muss in einer zweiten Matrix gespeichert werden um die Berechnungen der Nachbarzellen nicht zu verfälschen.
	
	Dieser Ansatz bietet den Vorteil, dass er mit OpenMP relativ einfach parallelisiert werden kann, denn die Berechnung jeder einzelnen Zelle ist unabhängig von den Berechnungen anderer Zellen.
	
	Besonderes Augenmerk muss man jedoch auf die Kanten und Ecken legen. Diese sollen, laut Aufgabenstellung, mit \verb|periodic boundary conditions| implementiert werden. Jedoch kann auch hier OpenMP zur Parallelisierung der Kanten verwendet werden.

	\subsection{Ein und Ausgabe}
	Da die Messung später in verschiedenen Feldgrößen durchgeführt wird, habe ich mich für den Einsatz von \verb|getopt| entschieden. Es ermöglicht die Anzahl der Schleifendurchläufe, die Feldgröße und eine optionale Fortschrittsanzeige über Argumente beim Programmstart einzustellen.
	
	\subsection{Probleme und ihre Lösung}
	Die Erzeugung des 2D-Arrays soll dynamisch erfolgen, damit man, ohne Anpassung des Programmcodes, die Feldgröße festlegen kann. Gemäß der Aufgabenstellung umfasst das größte Feld $32768\times 32768$ Zellen. Damit ist das Array zu groß für den Stack des Programms.
	Mein Lösungsansatz war die Allokation von Speicher mittels \verb|malloc| und \verb|double pointers|: Ein erstes Array wurde mit Pointern gefüllt, die jeweils wiederum auf die einzelnen Zeilen verweisen, welche ebenfalls mit \verb|malloc| allokiert wurden.
	
	\subsection{Optimierung des Codes}
	Eine Möglichkeit zur Einsparung von Rechenressourcen ist, den Speicherverbrauch des Programms zu reduzieren. So verwendete ich für die Zellen der Felder den Datentyp \verb|u_int8_t| statt \verb|int|. Dadurch reduziert sich der Speicherverbrauch jeder Zelle von 4 auf 1 Byte. Bei einem Feld der Größe $32768\times 32768$ entspricht dies einer stattlichen Einsparung von über 3 GiB.
		
	Des weiteren habe ich die \verb|double pointers| durch ein einzelnes \verb|malloc| ersetzt, indem die Anzahl von Spalten als Offset dient, um sich im 2D-Array zu bewegen. Dadurch vermeidet man einen Speicherzugriff und folglich eine Adressübersetzung bei jeder Datenabfrage und -manipulation.
\end{document}

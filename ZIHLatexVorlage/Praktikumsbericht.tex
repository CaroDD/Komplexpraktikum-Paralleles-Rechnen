\documentclass[german,plainarticle,hyperref,utf8]{zihpub}
\author{Daniel Körsten}
\title{Komplexpraktikum Paralleles Rechnen - Aufgabe B}
\matno{4690396}
\betreuer{Dr. Robert Schöne}


\begin{document}
	\section{Aufgabenbeschreibung}
	In dieser Aufgabe soll eine Thread-parallele Version von \texttt{Conway’s Game-of-Life} in der Programmiersprache \texttt{C} implementiert werden.
	
	Anschließend soll die Simulation mit verschieden großen Feldgrößen und Compiler durchgeführt und verglichen werden.
	
	\subsection{Conway’s Game-of-Life}
	 Das Game-of-Life ist ein vom Mathematiker John Horton Conway entworfenes Simulationsspiel \cite{gardner}. Es basiert auf einem zellulären Automaten. Häufig handelt es sich um ein Zweidimensionales Spielfeld, jedoch ist auch eine Dreidimensionale Simulation möglich.
	
	Das Spiel besteht dabei aus einem Feld mit einer festgelegten, möglichst großen, Anzahl an Zeilen und Spalten. Eine Zelle kann dabei entweder Tot oder Lebendig sein. Dieses Spielfeld wird mit einer zufälligen Anfangspopulation initialisiert.
	
	Ein Sonderfall stellen die Ecken und Kanten des Feldes dar, da dort nach den Spielregeln das Verhalten nicht festgelegt ist. Die Aufgabenstellung gibt vor, dass Spielfeld Torus-förmig sein soll. Alles was das Spielfeld auf einer Seite verlässt, kommt auf der gegenüberliegenden Seite wieder herein.
	
	Anschließend wird durch die Befolgung der Spielregeln die nächste Generation berechnet. Dafür betrachtet man jede Zelle und ihre 8 Nachbarn, um ihre Entwicklung zu berechnen. Es gelten folgende Spielregeln:
	\begin{enumerate}
		\item Eine lebende Zelle mit zwei oder drei Nachbarn überlebt in der Folgegeneration.
		\item Eine lebende Zelle mit vier oder mehr Nachbarn stirbt an der Überpopulation. Bei weniger als zwei Nachbarn stirbt sie an Einsamkeit.
		\item Jede tote Zelle mit genau drei Nachbarn wird in der nächsten Generation geboren.
	\end{enumerate}
	Wichtig ist, dass die Folgegenration für alle Zellen berechnet wird und anschließend die aktuelle Generation ersetzt. Es ist also nicht möglich die nachfolgende Generation im Spielfeld der Aktuellen zu berechnen.
	
	\subsection{Besonderheiten der Aufgabenstellung}
	Die Aufgabenstellung gibt vor, dass die Parallelisierung mittels \texttt{OpenMP} erfolgen soll. \texttt{OpenMP} ist eine API, welche es ermöglicht, Schleifen mithilfe von Threads zu parallelisieren \cite{openmp}. Es eignet sich hervorragend für \texttt{Shared-Memory Systeme}, also Systeme, bei denen mehrere Threads auf einen gemeinsamen Hauptspeicher zugreifen.
	
	Weitere Besonderheiten sind:
	\begin{itemize}
		\item Die Simulation soll variabel mit Feldgrößen von $128\times 128$ bis $32768\times 32768$ und 1 bis 32 Threads erfolgen.
		\item Das OpenMP Schedulingverfahren soll hinsichtlich des Einflusses auf die Ausführungszeit untersucht werden.
		\item Das Programm soll mit dem dem \texttt{GCC} und \texttt{ICC} kompiliert und anschließend getestet werden.
	\end{itemize}

	\section{Implementierung}
	 Zuerst habe ich mich mit der Abstraktion des Feldes in \texttt{C} beschäftigt. Meine Idee ist die Allokierung eines Speicherbereichs der Größe \texttt{columns * rows * sizeof(u\_int8\_t)} durch die C-Funktion \texttt{malloc()}. Innerhalb des Speicherbereichs kann man sich nun frei bewegen. Dabei verwendet man die \texttt{columns} als Offset um an die entsprechende Stelle zu springen.Beispiel: Möchte man auf die Zweite Zelle in Zweiten Zeile (da die Nummerierung typischerweise bei 0 beginnt, also das erste Element) zugreifen, würde man das \texttt{columns + 1} Byte innerhalb des Speicherbereichs verwenden.
	 
	 Der Datentyp \texttt{u\_int8\_t} benötigt dabei nur Ein Byte pro Zelle und ist für die Speicherung mehr als ausreichend, da ich nur den Zustand 0 - Zelle tot und 1 - Zelle lebendig speichern muss.
	 
	 Um zu Berücksichtigen, dass die Folgegeneration immer der aktuelle Generation ersetzt, allokiere ich einen zweiten Speicherbereich gleicher Größe. Vor dem Beginn einer neuen Berechnung, vertausche ich die beide Speicherbereiche, was dazu führt, dass die im vorhergehenden Schritt berechnete Folgegeneration zur aktuellen Generation wird und eine neue Generation berechnet werden kann.
	
	\subsection{Daten initialisieren}
	Gemäß den Startbedingungen muss nur eines der beiden Spielfelder mit Zufallswerten initialisiert werden.
	Um den Code möglichst einfach und effizient zu halten, verwende ich eine \texttt{for}-Schleife zur Iteration über jede Zelle des Arrays.
	
	Für die Dateninitialisierung jeder Zelle mit Null oder Eins, habe ich mich für Pseudo-Zufallszahlengenerator \texttt{rand\_r()} entschieden. Dieser ist, im Vergleich zu z.B. \texttt{rand()} Thread-sicher und kann Thread-parallel ausgeführt werden.
	
	Für die eigentliche Parallelisierung verwende ich die OpenMP Direktive:\\
	
	\texttt{\#pragma omp parallel for}\\
	
	Diese bewirkt, dass der Code innerhalb der Schleife parallel ausgeführt wird. \texttt{OpenMP} erzeugt bei betreten zusätzliche \textit{slave} Threads. Jeder bekommt einen Teil der Arbeit zugewiesen und führt diesen unabhängig von den anderen Threads aus. Wenn alle Threads ihre Arbeit erledigt haben, der parallel auszuführende Code also abgearbeitet wurde, fährt der \textit{master} Thread mit der seriellen Ausführung fort, bis er die nächste Direktive erreicht.
	seed bla bla\\
	
	\textbf{Anmerkung zu rand\_r():}
	
	\texttt{rand\_r()} wird in den \texttt{Linux Man Pages} als schwacher Pseudo-Zufallszahlengenerator geführt \cite{randr}. Das soll an dieser Stelle keine Relevanz haben, da der Spielverlauf und der Rechenaufwand nicht von der Güte des Zufallsgenerators abhängt.
		
	\subsection{Vorbereitung des Codes für eine parallele Ausführung}
	Für die Berechnung der nächsten Generation muss nun jede Zelle einzeln betrachtet werden und ihr Zustand in der nächsten Generation gemäß den Spielregeln berechnet werden.
	Ein möglicher Ansatz ist über jede Zeile und anschließend jede Spalte zu iterieren. Realisieren lässt sich das über zwei geschachtelte \verb|for|-Schleifen. Das Ergebnis dieser Berechnung muss in einer zweiten Matrix gespeichert werden um die Berechnungen der Nachbarzellen nicht zu verfälschen.
	
	Dieser Ansatz bietet den Vorteil, dass er mit OpenMP relativ einfach parallelisiert werden kann, denn die Berechnung jeder einzelnen Zelle ist unabhängig von den Berechnungen anderer Zellen.
	
	Besonderes Augenmerk muss man jedoch auf die Kanten und Ecken legen. Diese sollen, laut Aufgabenstellung, mit \verb|periodic boundary conditions| implementiert werden. Jedoch kann auch hier OpenMP zur Parallelisierung der Kanten verwendet werden.

	\subsection{Ein und Ausgabe}
	Da die Messung später in verschiedenen Feldgrößen durchgeführt wird, habe ich mich für den Einsatz von \verb|getopt| entschieden. Es ermöglicht die Anzahl der Schleifendurchläufe, die Feldgröße und eine optionale Fortschrittsanzeige über Argumente beim Programmstart einzustellen.
	Ebenso lassen sich Threadanzahl und OpenMP Schedulingverfahren einstellen.
	
	
\bibliography{Praktikumsbericht}
\end{document}
